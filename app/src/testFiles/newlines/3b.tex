\documentclass[fleqn,10pt]{article}
\usepackage{amsmath}
\usepackage{apacite}
\usepackage{color}
\usepackage{float}
\usepackage{graphicx}
\usepackage{listings}
\usepackage{spverbatim}
\usepackage{tabularx} % in the preamble
\begin{document}
\title{Analys av Dijkstra's Algoritm}

Det som var mest tidskrävande var att få rätt på hur man räknar ut kostnaden, samt hur man vänder en prioritets-kö!
Min IDE (Clion) visade ett fel, dock går kompilering bra, men fastnade där ett tag!
Annars gick det mesta rätt bra!
Rapporten tar tid och skriva, men det var förväntat, så inga problem där!

Vi skulle börjat lite tidigare!
Kom mycket i vägen, men det gick och lösa ändå!
Tycker att vi jobbade bra som ett team!
Jag var bortrest i början av projektet men vi lyckades fördela arbetet på ett bra sätt!
Implementationen hade varit mindre skökig om den hade använt pekare, då hade vi haft enklare att hitta våra structs, och hade inte behövt en getNode() funktion!
\end{document}
